\nonstopmode{}
\documentclass[a4paper]{book}
\usepackage[times,hyper]{Rd}
\usepackage{makeidx}
\usepackage[utf8,latin1]{inputenc}
% \usepackage{graphicx} % @USE GRAPHICX@
\makeindex{}
\begin{document}
\chapter*{}
\begin{center}
{\textbf{\huge Package `DO.db'}}
\par\bigskip{\large \today}
\end{center}
\begin{description}
\raggedright{}
\item[Title]\AsIs{A set of annotation maps describing the entire Disease Ontology}
\item[Description]\AsIs{A set of annotation maps describing the entire Disease Ontology assembled using data from DO}
\item[Version]\AsIs{2.9}
\item[Author]\AsIs{Jiang Li}
\item[Maintainer]\AsIs{Jiang Li }\email{riverlee2008@gmail.com}\AsIs{}
\item[Depends]\AsIs{R (>= 2.7.0), methods, AnnotationDbi (>= 1.9.7)}
\item[Imports]\AsIs{methods, AnnotationDbi}
\item[License]\AsIs{Artistic-2.0}
\item[biocViews]\AsIs{AnnotationData, FunctionalAnnotation}
\end{description}
\Rdcontents{\R{} topics documented:}
\inputencoding{utf8}
\HeaderA{DO.db}{Bioconductor annotation data package}{DO.db}
\aliasA{DO}{DO.db}{DO}
\keyword{datasets}{DO.db}
%
\begin{Description}\relax

Welcome to the DO.db annotation Package.  The purpose of this package
is to provide detailed information about the latest version of the
Disease Ontology.  This package is updated biannually.

You can learn what objects this package supports with the following command:

\code{ls("package:DO.db")} 

Each of these objects has their own manual page detailing where
relevant data was obtained along with some examples of how to use it.
\end{Description}
%
\begin{Examples}
\begin{ExampleCode}
  ls("package:DO.db")
\end{ExampleCode}
\end{Examples}
\inputencoding{utf8}
\HeaderA{DOANCESTOR}{Annotation of DO Identifiers to their Ancestors}{DOANCESTOR}
\keyword{datasets}{DOANCESTOR}
%
\begin{Description}\relax
This data set describes associations between DO 
terms and their ancestor  terms, based on the directed acyclic
graph (DAG) defined by the Disease Ontology Consortium. The format is an R
object mapping the DO  terms to all ancestor terms, where an
ancestor term is a more general DO term that precedes
the given DO term in the DAG (in other words, the parents, and all
their parents, etc.).
\end{Description}
%
\begin{Details}\relax
Each DO  term is mapped to a vector of ancestor DO  terms.



Mappings were based on data provided by: Disease Ontology
With a date stamp from the source of: 20150323 (sub\_version 2806)


\end{Details}
%
\begin{References}\relax
\url{http://do-wiki.nubic.northwestern.edu/index.php/Main_Page} 

\end{References}
%
\begin{Examples}
\begin{ExampleCode}
  # Convert the object to a list
  xx <- as.list(DOANCESTOR)
  # Remove DO IDs that do not have any ancestor
  xx <- xx[!is.na(xx)]
  if(length(xx) > 0){
    # Get the ancestor DO IDs for the first two elents of xx
    doids <- xx[1:2]
  }
  
\end{ExampleCode}
\end{Examples}
\inputencoding{utf8}
\HeaderA{DOCHILDREN}{Annotation of DO Identifiers to their Children}{DOCHILDREN}
\keyword{datasets}{DOCHILDREN}
%
\begin{Description}\relax
This data set describes associations between DO 
terms and their direct children  terms, based on the directed acyclic
graph (DAG) defined by the Disease Ontology Consortium. The format is an R
object mapping the DO  terms to all direct children terms, where a
direct child term is a more specific DO term that is immediately preceded
by the given DO term in the DAG.
\end{Description}
%
\begin{Details}\relax
Each DO  term is mapped to a vector of children DO  terms.


Mappings were based on data provided by: Disease Ontology
With a date stamp from the source of: 20150323 (sub\_version 2806)


\end{Details}
%
\begin{References}\relax
\url{http://do-wiki.nubic.northwestern.edu/index.php/Main_Page} 

\end{References}
%
\begin{Examples}
\begin{ExampleCode}
  # Convert the object to a list
  xx <- as.list(DOCHILDREN)
  # Remove DO IDs that do not have any children
  xx <- xx[!is.na(xx)]
  
  if(length(xx) > 0){
     # Get the parent DO IDs for the first elents of xx
        doids <- xx[[1]]
        # Find out the DO terms for the first parent doid
        DOID(DOTERM[[doids[1]]])
        Term(DOTERM[[doids[1]]])
        Synonym(DOTERM[[doids[1]]])
        Secondary(DOTERM[[doids[1]]])
  }
\end{ExampleCode}
\end{Examples}
\inputencoding{utf8}
\HeaderA{DOMAPCOUNTS}{Number of mapped keys for the maps in package DO.db}{DOMAPCOUNTS}
\keyword{datasets}{DOMAPCOUNTS}
%
\begin{Description}\relax
DOMAPCOUNTS provides the "map count" (i.e. the
count of mapped keys) for each map in package DO.db.
\end{Description}
%
\begin{Details}\relax
  
This "map count" information is precalculated and stored in the
package annotation DB. This allows some quality control and is used by
the \code{\LinkA{checkMAPCOUNTS}{checkMAPCOUNTS}}
function defined in AnnotationDbi to compare and validate different
methods (like \code{count.mappedkeys(x)} or
\code{sum(!is.na(as.list(x)))}) for getting the "map count" of a given
map.
\end{Details}
%
\begin{SeeAlso}\relax
\code{\LinkA{mappedkeys}{mappedkeys}},
\code{\LinkA{count.mappedkeys}{count.mappedkeys}},
\code{\LinkA{checkMAPCOUNTS}{checkMAPCOUNTS}}
\end{SeeAlso}
%
\begin{Examples}
\begin{ExampleCode}
  DOMAPCOUNTS
  mapnames <- names(DOMAPCOUNTS)
  DOMAPCOUNTS[mapnames[1]]
  x <- get(mapnames[1])
  sum(!is.na(as.list(x)))
  count.mappedkeys(x)   # much faster!

  ## Check the "map count" of all the maps in package DO.db
  #checkMAPCOUNTS("DO.db")
\end{ExampleCode}
\end{Examples}
\inputencoding{utf8}
\HeaderA{DOOBSOLETE}{Annotation of DO identifiers by terms defined by Disease Ontology Consortium and their status are obsolete}{DOOBSOLETE}
\keyword{datasets}{DOOBSOLETE}
%
\begin{Description}\relax
This is an R object mapping DO identifiers to the specific
terms in defined by Disease Ontology Consortium and their definition are obsolete
\end{Description}
%
\begin{Details}\relax
All the obsolete DO terms that are collected in this index will no longer exist 
in other mapping objects.

Mappings were based on data provided by: Disease Ontology
With a date stamp from the source of: 20150323 (sub\_version 2806)


\end{Details}
%
\begin{References}\relax
\url{http://do-wiki.nubic.northwestern.edu/index.php/Main_Page} 

\end{References}
%
\begin{Examples}
\begin{ExampleCode}
    # Convert the object to a list
    xx <- as.list(DOOBSOLETE)
    if(length(xx) > 0){
        # Get the TERMS for the first elent of xx
        DOID(xx[[1]])
        Term(xx[[1]])
    }
\end{ExampleCode}
\end{Examples}
\inputencoding{utf8}
\HeaderA{DOOFFSPRING}{Annotation of DO Identifiers to their Offspring}{DOOFFSPRING}
\keyword{datasets}{DOOFFSPRING}
%
\begin{Description}\relax
This data set describes associations between DO 
terms and their offspring  terms, based on the directed acyclic
graph (DAG) defined by the Disease Ontology Consortium. The format is an R
object mapping the DO  terms to all offspring terms, where an
ancestor term is a more specific DO term that is preceded
by the given DO term in the DAG (in other words, the children and all
their children, etc.).
\end{Description}
%
\begin{Details}\relax
Each DO  term is mapped to a vector of offspring DO  terms.


Mappings were based on data provided by: Disease Ontology
With a date stamp from the source of: 20150323 (sub\_version 2806)


\end{Details}
%
\begin{References}\relax
\url{http://do-wiki.nubic.northwestern.edu/index.php/Main_Page} 

\end{References}
%
\begin{Examples}
\begin{ExampleCode}
  # Convert the object to a list
  xx <- as.list(DOOFFSPRING)
  # Remove DO IDs that do not have any offspring
  xx <- xx[!is.na(xx)]
   if(length(xx) > 0){
    # Get the offspring DO identifiers for the first two elents of xx
    doidentifiers <- xx[1:2]
  }
\end{ExampleCode}
\end{Examples}
\inputencoding{utf8}
\HeaderA{DOPARENTS}{Annotation of DO Identifiers to their Parents}{DOPARENTS}
\keyword{datasets}{DOPARENTS}
%
\begin{Description}\relax
This data set describes associations between DO
terms and their direct parent  terms, based on the directed acyclic
graph (DAG) defined by the Disease Ontology Consortium. The format is an R
object mapping the DO  terms to all direct parent terms, where a
direct parent term is a more general DO term that immediately precedes
the given DO term in the DAG.
\end{Description}
%
\begin{Details}\relax
Each DO  term is mapped to a named vector of DO  terms. The name
associated with the parent term will be either \emph{isa}, \emph{partof},
where \emph{isa} indicates that the child term is a more specific version
of the parent, and  \emph{partof} indicate that the
child term is a part of the parent. 


Mappings were based on data provided by: Disease Ontology
With a date stamp from the source of: 20150323 (sub\_version 2806)


\end{Details}
%
\begin{References}\relax
\url{http://do-wiki.nubic.northwestern.edu/index.php/Main_Page} 

\end{References}
%
\begin{Examples}
\begin{ExampleCode}
  # Convert the object to a list
  xx <- as.list(DOPARENTS)
  # Remove DO IDs that do not have any parent
  xx <- xx[!is.na(xx)]
  if(length(xx) > 0){
     doids <- xx[[1]]
     # Find out the DO terms for the first parent do ID
     DOID(DOTERM[[doids[1]]])
     Term(DOTERM[[doids[1]]])
     Synonym(DOTERM[[doids[1]]])
     Secondary(DOTERM[[doids[1]]])
  }
\end{ExampleCode}
\end{Examples}
\inputencoding{utf8}
\HeaderA{DOSYNONYM}{Map from DO synonyms to DO terms}{DOSYNONYM}
\keyword{datasets}{DOSYNONYM}
%
\begin{Description}\relax
DOSYNONYM is an R object that provides mapping
from DO synonyms to DO terms
\end{Description}
%
\begin{Details}\relax
Mappings were based on data provided by: Disease Ontology
With a date stamp from the source of: 20150323 (sub\_version 2806)


\end{Details}
%
\begin{References}\relax
\url{http://do-wiki.nubic.northwestern.edu/index.php/Main_Page} 

\end{References}
%
\begin{Examples}
\begin{ExampleCode}
    x <- DOSYNONYM
    sample(x, 3)
    # DO ID "DOID:8757" has a lot of synonyms
    DOTERM[["DOID:8757"]]
    # DO ID "DOID:9134" is a synonym of DO ID "DOID:8757"
    DOID(DOSYNONYM[["DOID:9134"]])
\end{ExampleCode}
\end{Examples}
\inputencoding{utf8}
\HeaderA{DOTERM}{Annotation of DO Identifiers to DO Terms}{DOTERM}
\keyword{datasets}{DOTERM}
%
\begin{Description}\relax
This data set gives mappings between DO identifiers and their respective terms.
\end{Description}
%
\begin{Details}\relax
Each DO identifier is mapped to a \code{DOTerms} object that has 4 slots:
DOID: DO Identifier;
Term: The term for that DO id;
Secondary: Secondary terms that have been merged into this term;
Synonym: other  ontology terms that are considered to be synonymous to the primary
term attached to the DO id


Mappings were based on data provided by: Disease Ontology
With a date stamp from the source of: 20150323 (sub\_version 2806)


\end{Details}
%
\begin{References}\relax
\url{http://do-wiki.nubic.northwestern.edu/index.php/Main_Page} 

\end{References}
%
\begin{Examples}
\begin{ExampleCode}
    # Convert the object to a list
    FirstTenDOBimap <- DOTERM[1:10] ##grab the 1st ten
    xx <- as.list(FirstTenDOBimap)
     if(length(xx) > 0){
        # Get the TERMS for the 2nd element of xx
        DOID(xx[[2]])
        Term(xx[[2]])
        Synonym(xx[[2]])
        Secondary(xx[[2]])
    }
\end{ExampleCode}
\end{Examples}
\inputencoding{utf8}
\HeaderA{DOTerms-class}{Class "DOTerms"}{DOTerms.Rdash.class}
\aliasA{class:DOTerms}{DOTerms-class}{class:DOTerms}
\aliasA{DOID}{DOTerms-class}{DOID}
\aliasA{DOID,character-method}{DOTerms-class}{DOID,character.Rdash.method}
\aliasA{DOID,DOTerms-method}{DOTerms-class}{DOID,DOTerms.Rdash.method}
\aliasA{DOID,DOTermsAnnDbBimap-method}{DOTerms-class}{DOID,DOTermsAnnDbBimap.Rdash.method}
\aliasA{DOTerms}{DOTerms-class}{DOTerms}
\aliasA{initialize,DOTerms-method}{DOTerms-class}{initialize,DOTerms.Rdash.method}
\aliasA{Secondary}{DOTerms-class}{Secondary}
\aliasA{Secondary,character-method}{DOTerms-class}{Secondary,character.Rdash.method}
\aliasA{Secondary,DOTerms-method}{DOTerms-class}{Secondary,DOTerms.Rdash.method}
\aliasA{Secondary,DOTermsAnnDbBimap-method}{DOTerms-class}{Secondary,DOTermsAnnDbBimap.Rdash.method}
\aliasA{show,DOTerms-method}{DOTerms-class}{show,DOTerms.Rdash.method}
\aliasA{Synonym}{DOTerms-class}{Synonym}
\aliasA{Synonym,character-method}{DOTerms-class}{Synonym,character.Rdash.method}
\aliasA{Synonym,DOTerms-method}{DOTerms-class}{Synonym,DOTerms.Rdash.method}
\aliasA{Synonym,DOTermsAnnDbBimap-method}{DOTerms-class}{Synonym,DOTermsAnnDbBimap.Rdash.method}
\aliasA{Term}{DOTerms-class}{Term}
\aliasA{Term,character-method}{DOTerms-class}{Term,character.Rdash.method}
\aliasA{Term,DOTerms-method}{DOTerms-class}{Term,DOTerms.Rdash.method}
\aliasA{Term,DOTermsAnnDbBimap-method}{DOTerms-class}{Term,DOTermsAnnDbBimap.Rdash.method}
\keyword{methods}{DOTerms-class}
\keyword{classes}{DOTerms-class}
%
\begin{Description}\relax
A class to represent Disease Ontology nodes
\end{Description}
%
\begin{Section}{Objects from the Class}
Objects can be created by calls of the form
\code{DOTerms(DOId, term, synonym, secondary)}.
DOId, term are required.
\end{Section}
%
\begin{Section}{Slots}
\begin{description}

\item[\code{DOID}:] Object of class \code{"character"} A character
string for the DO id of a primary node.
\item[\code{Term}:] Object of class \code{"character"} A
character string that defines the role of gene product
corresponding to the primary DO id.
\item[\code{Synonym}:] Object of class \code{"character"} other
ontology terms that are considered to be synonymous to the primary
term attached to the DO id. Synonymous here can mean that the
synonym is an exact synonym of the primary term, is related to the
primary term, is broader than the primary term, is more precise
than the primary term, or name is related to the term, but is not
exact, broader or narrower.
\item[\code{Secondary}:] Object of class \code{"character"} DO ids
that are secondary to the primary DO id as results of merging DO
terms so that One DO id becomes the primary DO id and the rest
become the secondary.

\end{description}

\end{Section}
%
\begin{Section}{Methods}
\begin{description}

\item[DOID] \code{signature(object = "DOTerms")}:
The get method for slot DOID.
\item[Term] \code{signature(object = "DOTerms")}:
The get method for slot Term.
\item[Synonym] \code{signature(object = "DOTerms")}:
The get method for slot Synonym.
\item[Secondary] \code{signature(object = "DOTerms")}:
The get method for slot Secondary.
\item[show] \code{signature(x = "DOTerms")}:
The method for pretty print.

\end{description}

\end{Section}
%
\begin{Note}\relax
DOTerms objects are used to represent primary DO nodes in the
SQLite-based annotation data package DO.db
\end{Note}
%
\begin{References}\relax
\url{http://do-wiki.nubic.northwestern.edu/index.php/Main_Page} 
\end{References}
%
\begin{Examples}
\begin{ExampleCode}
  DOnode <- new("DOTerms", DOID="DOID:1234567", Term="Test")
  DOID(DOnode)
  Term(DOnode)

  ##Or you can just use these methods on a DOTermsAnnDbBimap
## Not run: ##I want to show an ex., but don't want to require DO.db
  require(DO.db)
  FirstTenDOBimap <- DOTERM[1:10] ##grab the 1st ten
  Term(FirstTenDOBimap)

  ##Or you can just use DO IDs directly
  ids = keys(FirstTenDOBimap)
  Term(ids)

## End(Not run)
\end{ExampleCode}
\end{Examples}
\inputencoding{utf8}
\HeaderA{DOTermsAnnDbBimap}{Class "DOTermsAnnDbBimap" }{DOTermsAnnDbBimap}
\aliasA{class:DOTermsAnnDbBimap}{DOTermsAnnDbBimap}{class:DOTermsAnnDbBimap}
\aliasA{DOTermsAnnDbBimap-class}{DOTermsAnnDbBimap}{DOTermsAnnDbBimap.Rdash.class}
\keyword{classes}{DOTermsAnnDbBimap}
\keyword{interface}{DOTermsAnnDbBimap}
%
\begin{Description}\relax
A sub-class of  Bimap, specific for DO.db. Please see \code{\LinkA{Bimap}{Bimap}} for details.
\end{Description}
%
\begin{SeeAlso}\relax
\code{\LinkA{Bimap}{Bimap}},
\code{\LinkA{DOTerms}{DOTerms}}
\end{SeeAlso}
%
\begin{Examples}
\begin{ExampleCode}
   class(DOTERM)
\end{ExampleCode}
\end{Examples}
\inputencoding{utf8}
\HeaderA{DO\_dbconn}{Collect information about the package annotation DB}{DO.Rul.dbconn}
\aliasA{DO\_dbfile}{DO\_dbconn}{DO.Rul.dbfile}
\aliasA{DO\_dbInfo}{DO\_dbconn}{DO.Rul.dbInfo}
\aliasA{DO\_dbschema}{DO\_dbconn}{DO.Rul.dbschema}
\keyword{utilities}{DO\_dbconn}
\keyword{datasets}{DO\_dbconn}
%
\begin{Description}\relax
Some convenience functions for getting a connection object to (or collecting
information about) the package annotation DB.
\end{Description}
%
\begin{Usage}
\begin{verbatim}
  DO_dbconn()
  DO_dbfile()

  DO_dbschema()
  DO_dbInfo()
\end{verbatim}
\end{Usage}
%
\begin{Details}\relax
\code{DO\_dbconn} returns a connection object to the
package annotation DB.  IMPORTANT: Don't call
\code{\LinkA{dbDisconnect}{dbDisconnect}} on the connection object
returned by \code{DO\_dbconn} or you will break all the
\code{\LinkA{AnnDbObj}{AnnDbObj}} objects defined
in this package!

\code{DO\_dbfile} returns the path (character string) to the
package annotation DB (this is an SQLite file).

\code{DO\_dbschema} prints the schema definition of the
package annotation DB.

\code{DO\_dbInfo} prints other information about the package
annotation DB.
\end{Details}
%
\begin{Value}
\code{DO\_dbconn}: a DBIConnection object representing an
open connection to the package annotation DB.

\code{DO\_dbfile}: a character string with the path to the
package annotation DB.

\code{DO\_dbschema}: none (invisible \code{NULL}).

\code{DO\_dbInfo}: none (invisible \code{NULL}).
\end{Value}
%
\begin{SeeAlso}\relax
\code{\LinkA{dbGetQuery}{dbGetQuery}},
\code{\LinkA{dbConnect}{dbConnect}},
\code{\LinkA{dbconn}{dbconn}},
\code{\LinkA{dbfile}{dbfile}},
\code{\LinkA{dbschema}{dbschema}},
\code{\LinkA{dbInfo}{dbInfo}}
\end{SeeAlso}
%
\begin{Examples}
\begin{ExampleCode}
  ## Count the number of rows in the "do_term" table:
  dbGetQuery(DO_dbconn(), "SELECT COUNT(*) FROM do_term")

  ## The connection object returned by DO_dbconn() was
  ## created with:
  dbConnect(SQLite(), dbname=DO_dbfile(), cache_size=64000,
  synchronous=0)

  #DO_dbschema()

#  DO_dbInfo()
\end{ExampleCode}
\end{Examples}
\printindex{}
\end{document}
